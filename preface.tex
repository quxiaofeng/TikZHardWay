\section*{前言:笨办法更简单} % (fold)
\label{sec:前言_笨办法更简单}
\addcontentsline{toc}{section}{前言:笨办法更简单}
\markright{前言:笨办法更简单}
这本小册子的目的,是让你在\TikZ 绘图上入门。虽然名称是“笨办法”,但其实并非如此。所谓的“笨办法”只是本文的行文方式,内容并不笨。在本文的帮助下,你将通过许多简单的练习,学会\TikZ 这一绘图语言。而做练习,是学习任何类编程语言的必经之路:

\begin{compactitem}
	\item 做每一道习题;
	\item 一字不差地写出每一段代码;
	\item 编译代码,让它得到正确的结果。
\end{compactitem}

在刚开始,做到这些对你来说会非常难,但你需要坚持下去。如果你每晚都花费一两个小时去做这里的练习,你除了将会学会\TikZ 之外,还会了解到学习类编程语言的一般方法。因为对于新手来说三种最重要的技能是:读和写、注重细节、发现不同。

\subsection*{不要复制粘贴} % (fold)
\label{sub:不要复制粘贴}
你必须手动将每个练习亲自输入进电脑,而不是复制文中的代码——复制粘贴会让这些练习变得毫无意义。这些习题的目的是训练你的双手和大脑思维,让你有能力读代码、写代码、观察代码。如果你复制粘贴的话,那你就是在欺骗自己,而且这些练习的效果也将大打折扣。
% subsection 不要复制粘贴 (end)

\subsection*{灵感来源} % (fold)
\label{sub:灵感来源}
这本《笨办法》的灵感来源于Zed~A. Shaw先生写的《Learn Python The Hard Way》。
% subsection 灵感来源 (end)

\subsection*{许可协议} % (fold)
\label{sub:许可协议}
你可以在不收取任何费用,而且不修改任何内容的前提下自由分发这本小册子给任何人。但是本书的内容只允许完整原封不动地进行分发和传播。也就是说,如果你使用这本小册子给人讲课是可以的,但是不允许向他们收费。

Copyright \textcopyright~2013 by Liam Huang.
% subsection 许可协议 (end)
% section 前言_笨办法更简单 (end)