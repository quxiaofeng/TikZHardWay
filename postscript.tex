\section*{下一步} % (fold)
\label{sec:下一步}
\addcontentsline{toc}{section}{下一步}
\markright{下一步}
现在还不能说你是一个程序员。这本书的目的相当于给你一个“编程棕带”。你已经了解了足够的编程基础,并且有能力阅读别的编程书籍了。读完这本书,你应该已经掌握了一些学习的方法,并且具备了该有的学习态度,这样你在阅读其他 Python 书籍时也许会更顺利,而且能学到更多东西。

在 \url{http://learnpythonthehardway.org/} 网站列出了一些你可以进一步阅读的免费书籍,试着阅读它们,看看自己可以走多远。

或许,你现在已经可以开始鼓捣一些程序出来了。如果你手上有需要解决的问题,试着写个程序解决一下。你一开始写的东西可能很挫,不过这没有关系。以我为例,我在学每一种语言的初期都是很挫的。没有哪个初学者能写出完美的代码来,如果有人告诉你他有这本事,那他只是在厚着脸皮撒谎而已。

最后,记住学习编程是要投入时间的,你可能需要至少每天晚上练习几个小时。顺便告诉你,当你每晚学习 Python 的时候,我在努力学习弹吉他。我每天练习2 到 4 小时,而且还在学习基本的音阶。

每个人都是某一方面的菜鸟。
% section 下一步 (end)
\section*{老程序员的建议} % (fold)
\label{sec:老程序员的建议}
\addcontentsline{toc}{section}{老程序员的建议}
\markright{老程序员的建议}
你已经完成了这本书而且打算继续编程。也许这会成为你的一门职业,也许你只是作为业余爱好玩玩。无论如何,你都需要一些建议以保证你在正确的道路上继续前行,并且让这项新的爱好为你带来最大程度的享受。

我从事编程已经太长时间,长到对我来说编程已经是非常乏味的事情了。我写这本书的时候,已经懂得大约 20 种编程语言,而且可以在大约一天或者一个星期内学会一门编程语言(取决于这门语言有多古怪)。现在对我来说编程这件事情已经很无聊,已经谈不上什么兴趣了。

在这么久的旅程下来我的体会是:编程语言这东西并不重要,重要的是你用这些语言做的事情。事实上我一直知道这一点,不过以前我会周期性地被各种编程语言分神而忘记了这一点。现在我是永远不会忘记这一点了,你也不应该忘记这一点。

你学到和用到的编程语言并不重要。不要被围绕某一种语言的宗教把你扯进去,这只会让你忘掉了语言的真正目的,也就是作为你的工具来实现有趣的事情。

编程作为一项智力活动,是唯一一种能让你创建交互式艺术的艺术形式。你可以创建项目让别人使用,而且你可以间接地和使用者沟通。没有其他的艺术形式能做到如此程度的交互性。电影领着观众走向一个方向,绘画是不会动的。而代码却是双向互动的。

编程作为一项职业只是一般般有趣而已。编程可能是一份好工作,但如果你想赚更多的钱而且过得更快乐,你其实开一间快餐分店就可以了。你最好的选择是将你的编程技术作为你其他职业的秘密武器。

技术公司里边会编程的人多到一毛钱一打,根本得不到什么尊敬。而在生物学、医药学、政府部门、社会学、物理学、数学等行业领域从事编程的人就能得到足够的尊敬,而且你可以使用这项技能在这些领域做出令人惊异的成就。

当然,所有的这些建议都是没啥意义的。如果你跟着这本书学习写软件而且觉得很喜欢这件事情的话,那你完全可以将其当作一门职业去追求。你应该继续深入拓展这个近五十年来极少有人探索过的奇异而美妙的智力工作领域。若能从中得到乐趣当然就更好了。

最后我要说的是学习创造软件的过程会改变你而让你与众不同。不是说更好或更坏,只是不同了。你也许会发现因为你会写软件而人们对你的态度有些怪异,也许会用“怪人”这样的词来形容你。也许你会发现因为你会戳穿他们的逻辑漏洞而他们开始讨厌和你争辩。甚至你可能会发现有人因为你懂得计算机怎么工作而觉得你是个讨厌的怪人。

对于这些我只有一个建议: 让他们去死吧。这个世界需要更多的怪人,他们知道东西是怎么工作的而且喜欢找到答案。当他们那样对你时,只要记住这是你的旅程,不是他们的。“与众不同”不是谁的错,告诉你“与众不同是一种错”的人只是嫉妒你掌握了他们做梦都不能想到的技能而已。

你会编程。他们不会。这真他妈的酷。
% section 老程序员的建议 (end)